\section{Details of a package}

\subsection{The package directory}

We define Package (version 1.0) a directory with the following characteristics

\begin{itemize}
\item it has the .package extension
\item removed the extension, it has a specific name in the form package\_name-version\_major.version\_minor.patch\_level (ex : foo-1.0.0)
\item it contains an xml file called manifest.xml
\item it optionally contains subdirs to host executables, resources and other.
\end{itemize}

We also defined the following nomenclature:
\begin{itemize}
\item an \kw{executable} is a binary or script that can be executed on a given platform, as defined by `uname`-`uname -m`
\item a \kw{resource} is a file that is meaningful on a given platform, as defined as before
\item an \kw{executable group} is a collection of platform-dependent executables. The executable group can be referred by means of a single, unique label (\kw{entry point}).
\item an \kw{resource group} is a collection of platform-dependent resources. The resource group can be referred by a single, unique label (entry point).
\end{itemize}

\subsection{The manifest}

the manifest.xml is an xml file describing the package. It says which
executables and resources are contained in the package, and their paths.
Different entry\_points are assigned to each executable and resource groups.

In a meta section, it is possible to define a default entry\_point that
represents the package. If a default entry point is defined, the package itself
is said to be runnable. The entry point must refer to an executable group.
entry points must be unique within a package.

\begin{verbatim}
<?xml version="1.0" encoding="UTF-8"?>
<Package xmlns="urn:uuid:d195be0c-200a-40a4-9d05-35fdf42eb29f" version="1.0.0">
    <Meta>
        <DefaultExecutableGroupEntryPoint>defaultExecutable</DefaultExecutableGroupEntryPoint>
        <Description>Package description</Description>
    </Meta>
    <Contents>
        <ExecutableGroup entryPoint="defaultExecutable">
            <Executable>
                <Platform>noarch</Platform>
                <Interpreter>bash</Interpreter>
                <Path type="standard">defaultExecutable</Path>
            </Executable>
        </ExecutableGroup>
        <ExecutableGroup entryPoint="secondaryExecutable">
            <Executable>
                <Platform>noarch</Platform>
                <Interpreter>python</Interpreter>
                <Path type="standard">secondaryExecutable</Path>
            </Executable>
        </ExecutableGroup>
        <ResourceGroup entryPoint="resource1">
            <Resource>
                <Platform>Linux-ia64</Platform>
                <Path type="standard">resource1</Path>
            </Resource>
        </ResourceGroup>
        <ResourceGroup entryPoint="resource2" description="a description">
            <Resource>
                <Platform>noarch</Platform>
                <Path type="standard">resource2</Path>
            </Resource>
        </ResourceGroup>
    </Contents>
</Package>
\end{verbatim}

The namespace is the uuid \verb+d195be0c-200a-40a4-9d05-35fdf42eb29f+. It has
been generated randomly.  

The Meta section (optional) contains the
DefaultExecutableGroupEntryPoint (optional) for the package. The entry point name
specified as default entry point must also be defined for an executable group,
otherwise the package manifest is not correct, and the package is not a
package. It can also contain a free-text Description (optional) for the package.

The Contents section (mandatory) describes the actual contents of the package. There are two
possible subelements to the Content element: ExecutableGroup and ResourceGroup.
both of them have a single mandatory attribute which is the entry point name, a
unique name identifying it. If some entry point is not specified, the entry is
considered corrupted and therefore discarded.

The ExecutableGroup element contains Executable elements, one for each platform.
This way it is possible to have multiple architectures into the same package,
and the correct one will be ran. The Path element has a single attribute
"type", which can be one of the following:
\begin{itemize}
\item standard: means that the path is relative to the package subdirectory "Executables/platform\_string" (ie: Executables/Linux-ia64)
\item package\_relative: means that the path is relative to the root of the package.
\item absolute: means that the path is an absolute path from the beginning of
      the filesystem. Useful to define "interface" packages to see as packages 
      applications that already reside on the disk somewhere and cannot be moved or packaged.
\end{itemize}

At the same way, ResourceGroup contain Resource elements, each one specific for a given platform.

\subsection{Package layout}

Executables are normally stored into "Executables/platform/". Resource are normally stored into "Resources/platform/".
There is no standardization for anything else, however in general you should use "Libraries/platform/language" for internal libraries,
and "Documentation" for documentation.

\subsection{Programs}

An executable called cnrun must be reachable in the PATH variable. When
invoked with the package name, the variable PACKAGE\_SEARCH\_PATH is walked in
a similar way to PATH to find the package.  If the package is found and is
runnable, the executable is executed. When executed, the variable
PACKAGE\_ROOT\_DIR is defined, containing the absolute path to the package root
directory.  It is possible to specify a different entry point as
"Package-1.0.0/entry\_point".

cnrun always tries to run the platform specific executable if available. If
this fails, it falls back to a noarch version if available.

A similar executable, called cnpath, gets the resource absolute path out of the
package and resource entry point.  Like before, the entry point is specified
with a slash (for example cnpath package-1.0.0/resourcename)

It is also possible to specify a package without the version: in this case, the
most recent version will be chosen by pure numeric criterium.  For example. having the following packages in our search path
\begin{verbatim}
foo-1.0.0
foo-1.0.1
foo-1.1.0
foo-1.2.0
foo-2.1.0
\end{verbatim}

calling cnrun foo will run foo-2.1.0; Calling cnrun foo-1 will run
foo-1.2.0; Calling cnrun foo-1.0 will run foo-1.0.1

The executable cnls produces a list of all the packages found while scouting the search path.
