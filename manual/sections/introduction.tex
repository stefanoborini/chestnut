\section{Introduction}

Welcome to the documentation of Chestnut Package Manager, version \version{}.
Chestnut is a utility that allows you to organize programs and resource files
in self-contained packages you can move transparently\footnote{conditions
apply. See the rest of this documentation for more details} around in your
filesystem (or even on different computer architectures).

Quite often, a huge number of utilities and scripts can be present on our
computers, each one performing from small to complex tasks.  All these
executables reside somewhere on the local disk, or on an NFS shared disk.
Referring to them via absolute or relative paths could introduce problems if
you want to move them around. Moreover, there can be different versions of the same
program, or the same program must be made available for multiple architectures.
In some cases, different executables can be part of the same software suite,
and it would be nice to have them accessible through a single entity. Finally,
in some cases the previous condition does not hold only for executable programs, but also for data resources.

We want to address the following issues, and we want to do it easily:

\begin{itemize}
\item \textbf{Versioning}: know which version of a given executable we are running, or resource
we are getting
\item \textbf{Path independence}: we don't want to depend on static paths to
invoke a given executable, or to get a specific resource.
\item \textbf{Encapsulation}: keep related executables and resources
together as much as we can, potentially with their private helper libraries
into a single, independent and fully contained entity.
\item \textbf{Relocatability}: we want to be able to refer to both executables
and data with a simple name, and let the system figure out where they are, when
they are needed. If our targets are relocated, the lookup system should be easy
to reconfigure to the new location, and no other intervention will be needed to
restore the behavior of scripts invoking these executables.
\item \textbf{Platform transparency}: we should be able to store many platform-dependent
copies of an executable, and let the system figure out the correct one to invoke.
\end{itemize}

We don't want to use paths because they force us to keep a specific location
for a given utility. To solve this problem, a possible solution is to rely on
the famous PATH variable the shell uses to resolve the executable names.
Although clever, it suffers from some drawbacks: 
\begin{enumerate}
\item it is difficult to do versioning (unless we name our executable properly)
and to keep track of what is running.
\item in order to make hundreds of scripts and programs accessible via PATH
search, we need to put all the entries into the variable making it very long
and difficult to manipulate
\item PATH works only for executables. We would like to access data
resources as well with the same mechanism.
\end{enumerate}

Chestnut is similar in some way to Application packages in NextStep/MacOSX
and Java Archives. Executables, resources, and their libraries are put into a
directory with a special name. A program, \textbf{cnrun} can be invoked to run
a program packaged in this way. 
