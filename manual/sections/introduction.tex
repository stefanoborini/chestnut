\section{Introduction}

\subsection{Problem to solve}

Quite often, a huge number of utilities and scripts can be present on our
computers, each one performing from small to complex tasks. All these
executables and scripts reside somewhere on the disk, have different version
numbers and sometimes they are incompatible due to change in input/output.

Also, these executables normally need to access resources, like static
input files containing well known data.  It is a good idea to keep
the executables and the files used by these executables together as much as
possible, or to eventually collect the resources together so that they have
a version.

It is important to get access to these executables and resources, while at
the same time guaranteeing the following points:

\begin{itemize}
\item \textbf{Versioning}: know which version of a given executable we are running, or resource
we are getting
\item \textbf{Path independence}: we don't want to depend on static paths to
invoke a given executable, or to get a specific resource.
\item \textbf{Encapsulation}: keep related executables and resources
together as much as we can, potentially with their private helper libraries
into a single, independent entity.
\end{itemize}

We don't want to use static paths because they force us to keep a
specific location for a given utility. To solve this problem, a possible
solution is to rely on the famous PATH variable the shell uses to resolve
the executable names. This is what (among other things) the first
PackageManager by Tim Burrell. Although clever, it suffers from some
drawbacks: 
\begin{enumerate}
\item it is difficult to do versioning (unless we name our executable properly)
and to keep track of what is running.
\item in order to make hundreds of scripts and programs accessible via PATH
search, we need to put all the entries into the variable making it very long
and difficult to manipulate
\item PATH searches only for executables. We would like to access resources
as well.
\end{itemize}

A different approach is needed. 

\subsection{Solution}

The solution implemented here is an hybrid/stripped down rehack between
technologically two solid solutions: Application packages in NextStep/MacOSX
and java archives.

Executables, resources, and their libraries are put into a directory with a
special name. A program, \textbf{cnrun} can be invoked to run a program
packaged in this way. 

