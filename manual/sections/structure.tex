\section{Structure}

Chestnut Package manager is composed of the following programs:
\begin{itemize}
\item \textbf{cnrun}: runs a package
\item \textbf{cnpath}: gives the absolute path of a resource or executable contained into a
package
\item \textbf{cnls}: gives a list of all the packages found, or print
more detailed informations about a specific package.
\end{itemize}

Packages are special directories with a special content. A tutorial on how
to create and setup your own package will be provided as additional technical
information, but it is not in the scope of the present document.

Packages are stored in a shared repository. The advantage of a package is
that it describes itself, it is relocatable, and promotes self-contained
programs. In general, packages repositories are one or more directories we
designate as such, plus a personal repository in your home directory. If you
declare some of your personal packages stable enough, in order to make them
available to the others is sufficient to copy them in the shared repository.
