\section{Package structure}

\subsection{The package layout}

Chestnut Packages can be either 
\begin{enumerate}
\item special directories with a special name, content and structure.
\item a zipfile with a proper name, containing the same data as the above directory (from Chestnut version 2.2.0) 
\end{enumerate}

The directory name must be in the form \verb+name-ver_major.ver_minor.patchlevel.chestnut+.
As of version 2.2.0, the preferred extension for directory-based packages is
\verb+.chestnut+, superseding the \verb+.package+ which is kept for compatibility.
This means that all of the following are valid package names in Chestnut 2.2.0
\begin{verbatim}
foo-1.2.3.package/
MyPackage-0.1.3.chestnut/
MyPackage-4.0.0.chestnut/
\end{verbatim}
while these are not
\begin{verbatim}
foo-1.2.package/              # no patchlevel
hello-0.2.beta.chestnut/      # string in patchlevel
MyPackage-4.0.0.foo/          # invalid extension
MyPackage-4.0.0.1.chestnut/   # invalid version structure
-4.0.0.chestnut/              # empty package name
\end{verbatim}

The package directory must contain a xml file called \verb+manifest.xml+.
Standard directories for executables and resources are optional, and the
standard layout is \verb+Executables/architecture-string/+ and
\verb+Resources/architecture-string/+.
An example of package layout:
\begin{verbatim}
foo-1.2.3.chestnut/
foo-1.2.3.chestnut/manifest.xml
foo-1.2.3.chestnut/Executables/Linux-i686/program
foo-1.2.3.chestnut/Executables/Darwin-i386/program
foo-1.2.3.chestnut/Executables/noarch/program
foo-1.2.3.chestnut/Resources/Linux-i686/file
foo-1.2.3.chestnut/Resources/noarch/file
\end{verbatim}

Any other less-official directory schema can be created if useful. For example,
the author finds convenient to store shared libraries in
Libraries/architecture/language/, but at the moment no standard is suggested on
this regard.

Starting from Chestnut 2.2.0, you can create a zip archive (with extension
\verb+.nutz+) of the contents of a \verb+.chestnut+ directory.  The zip file
must contain the manifest.xml file at the root. It can be created with the
following steps
\begin{verbatim}
$ cd MyPackage-1.2.0.chestnut
$ zip -r ../MyPackage-1.2.0.nutz .
\end{verbatim}
the package can then be deployed in the usual way.


\subsection{The manifest}
The package describes itself through an XML file called the manifest.
A manifest.xml file in the base directory of the package describes the package
contents and meta information about executables, resources and so on. It
explicitly exposes the public interface to the internals of the package.

The current manifest file version 1.0 has a Meta section, and a Contents section. 
The Meta section is optional and contains metainformation about the package as a whole.

\begin{verbatim}
<?xml version="1.0" encoding="UTF-8"?>
<Package xmlns="urn:uuid:d195be0c-200a-40a4-9d05-35fdf42eb29f" version="1.0.0">
    <Meta>...</Meta>
    <Contents>...</Contents>
</Package>
\end{verbatim}

The Contents section is mandatory. It hosts two types of xml elements:
ExecutableGroup and ResourceGroup.  They are almost similar, except that one
defines Executable entry points, and the other Resource entry points.

\begin{verbatim}
    <Contents>
        <ExecutableGroup entryPoint="hello" 
                         description="Prints hello in the specified language">
            <Executable>
                <Platform>Darwin-Power Macintosh</Platform>
                <Interpreter>bash</Interpreter>
                <Path type="standard">hello</Path>
            </Executable>
            <Executable>
                <Platform>noarch</Platform>
                <Interpreter>bash</Interpreter>
                <Path type="standard">hello</Path>
            </Executable>
        </ExecutableGroup>
    </Contents>
\end{verbatim}
Every executable group must have the entryPoint attribute defined as a unique
string.  An optional description can be specified as well, and will be printed
when \verb+cnls -l+ is invoked. 

An executable group contains Executable entries, one for each platform. This
allows to specify the same program compiled for different architectures.
Chestnut will automatically choose the proper executable, depending on the
architecture of the machine it is running on. If a compatible entry for the
architecture is not found, the "noarch" platform is tried, if available.  This
feature is very useful if you have a NFS mounted Package repository, with
different architectures accessing this NFS. When the entry point is executed,
the proper executable will be used.

Each executable entry must specify a Platform and a Path. Interpreter is
optional, and it's the interpreter which will be used for scripts. 
\footnote{As of version 2.2, the interpreter must not contain a path to the interpreter,
just the name. It is assumed that the interpreter can be found on the system
through regular PATH. It is also not possible to specify options. This
restriction will be removed in 2.3}
The Platform must contain the result of the command \verb+echo `uname`-`uname -m`+, or the string \verb+noarch+
The Path specifies the location of the associated executable. It has a mandatory attribute type which can have one
of the following values

\begin{enumerate}
\item \verb+standard+: the specified executable name is searched into the standard package
subdirectory “Executables/`uname`-`uname -m`/” (eg. “Executables/Linux-i686/”).
\item \verb+package_relative+: the executable is searched under the package
subdirectory, but the actual relative path is up to you to decide.
\item \verb+absolute+: the executable can be anywhere on the filesystem.  The full,
absolute path must be provided. Useful to create “virtual packages”, and let
unpackaged programs be accessible through the Chestnut interface 
\end{enumerate}

For an executable to be properly recognized, the following conditions must be met:
\begin{enumerate}
\item the executable must exists and accessible at the specified path
\item the executable must be executable (\verb=+x=)
\end{enumerate}
or
\begin{enumerate}
\item the executable must exists and accessible at the specified path
\item the interpreter must be available and executable
\end{enumerate}

If any of these conditions does not hold, the entry will not be available.

The exact same structure exists for ResourceGroups. Resource section does not accept Interpreter of course.
A platform must be specified for resources as well (some data can be platform
dependent, like in case of endianness issues, or with data files optimized for
a specific platform). In general, however, the noarch platform will be used.

\subsection{Additional features of the manifest}
\subsubsection{default entry point}

In the manifest Meta section, it is possible to specify a default executable group to be executed.

\begin{verbatim}
<Meta>
      <DefaultExecutableGroupEntryPoint>hello</DefaultExecutableGroupEntryPoint>
</Meta>
\end{verbatim}

Of course, the executable group entry point must exist.

\subsubsection{deprecation of entry points and packages}
It is possible to document an entry point as deprecated with the \verb+deprecated+ attribute
\begin{verbatim}
<Contents>
      <ExecutableGroup entryPoint="hello" deprecated="deprecation message">
<...>
\end{verbatim}

It is also possible to document a whole package as deprecated with the Deprecated tag in the Meta section
\begin{verbatim}
<Meta>
    <Deprecated>deprecation message</Deprecated>
<...>
\end{verbatim}

\subsection{The content}

In order to create a functional package, the application must not rely
on specific hardcoded paths at compile time to get its static
resources (eg. like icons, template documents, default configuration files
etc). To get the current package location you can get the environment variable
\verb+CN_ROOT_DIR+. Chestnut sets this variable before running the package to
the current absolute path of the package directory, so when this variable is
set, you can conclude that the application is running in a packaged
environment. If the package is moved somewhere else, the value will be the
proper one for the new location.

Applications that do not access other files, like applications getting input
from stdin, or that require the file to access through a command line argument
are fully relocatable.

For libraries, applications using system libraries will run flawlessly, as the
dynamic linker will do the job no matter where the application is. If the
application is using private libraries stored into the package, and some
environment variable must be exported to make them accessible, you can write a
wrapper script that first export these variables (eg. in \verb+LD_LIBRARY_PATH+
or \verb+DYLD_LIBRARY_PATH+), and then runs the application.

When any entry point in a chestnut package is invoked, the following variables are defined.
\begin{itemize}
\item \verb+CN_ROOT_DIR+: contains the absolute path of the chestnut package root directory. Supersedes \verb+PACKAGE_ROOT_DIR+, which is kept for compatibility, but deprecated.
\item \verb+CN_ENTRY_POINT+: contains the entry point string
\item \verb+CN_RUN_ARCH+: contains the actual used architecture from the executable group.
\end{itemize}



