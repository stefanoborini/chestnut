\section{Usage}

\subsection{Getting a list of the available packages with cnls}

The first idea to get accustomed with the system is to run cnls. It
gives a list of all the packages it can find, in the order they are found.

\begin{verbatim}
$ cnls
testPackage-1.0.0
roseplot-1.1.0
saco_utils-0.7.5
\end{verbatim}

You can also use globbing, but remember that you need to escape the symbols that could be interpreted by the shell.
\begin{verbatim}
$ cnls \*plot\*
roseplot-1.1.0
\end{verbatim}

you can obtain more detailed informations if you specify the option
\verb+-l+

\begin{verbatim}
testPackage-1.0.0:
    Flags: xrdm
    Location: 
        <note: splitted for readability here in the manual>
        /Users/stefano/Work/GenomeAtlas/runApp/tests/
            testPackageDir1/testPackage-1.0.0.package/
    Executable entry points:
        defaultExecutable (default)
        secondaryExecutable 
    Resource entry points:
        resource1
        resource2
\end{verbatim}

A package has \textbf{entry points}: an entry point is a name that you can
add to a package in order to get something specific contained into that
package. We'll go back to them later.

The location is where the package was found.

The flags specify informations about the package. The presence of:
\begin{itemize}
\item an \verb+x+ means that the package contains executables
\item a \verb+r+ means that the package contains resources
\item a \verb+d+ means that the package has a default entry point for an
executable. This has consequences when you run it with cg\_run.
\item a \verb+m+ means that the package has executable entry points
\textbf{other} than the default. With "other" we mean that if there's no
default, this flag is present also if you have only one executable entry
point.
\end{itemize}

A package and each entry point can contain brief descriptions about their task.
You can print the brief list of packages together with their description by using the option
\verb+-d+.

\subsection{Running a package with cnrun}

You can run a package with the command \verb+cnrun+.
If a package has a default entry point, the only thing you have to specify
is the package name (with version, in this example):
\begin{verbatim}
$ cnrun testPackage-1.0.0
The default executable was executed
\end{verbatim}

you can eventually run an alternative entry point with this syntax
\begin{verbatim}
$ cnrun testPackage-1.0.0/secondaryExecutable
The secondary executable was executed
\end{verbatim}

this is therefore totally equivalent to the first example
\begin{verbatim}
$ cnrun testPackage-1.0.0/defaultExecutable
The default executable was executed
\end{verbatim}

It is also possible to specify a package without the version: in this case,
the most recent version will be chosen by lexicographic criterium.  For
example. having the following packages in our search path
\begin{verbatim}
foo-1.0.0
foo-1.0.1
foo-1.1.0
foo-1.2.0
foo-2.1.0
\end{verbatim}

\begin{itemize}
\item calling \verb+cnrun foo+ will run foo-2.1.0
\item calling \verb+cnrun foo-1+ will run foo-1.2.0
\item calling \verb+cnrun foo-1.0+ will run foo-1.0.1
\end{itemize}

If you have multiple packages with the same name, the one that gets executed
is the first that is found (honors the order specified in
PACKAGE\_SEARCH\_PATH). 

cnrun is totally transparent in terms of input/output. you can use it in
unix pipelines, redirect the stdin or stdout, and pass options to the
invoked program. You can even run standard unix programs (although there's
apparently no real reason to do it, it is convenient to do so in some
advanced circumstances). If the package is not found, cnrun will try to
execute the program as in a standard shell PATH lookup. In other words, you
can do this
\begin{verbatim}
$ cnrun ls -la | cnrun less
\end{verbatim}

\subsection{Getting the path of what you need: cnpath}

The program cnpath gets the absolute path out of the package
and resource entry point. In the case of resources, specifying the entry
point is mandatory.

\begin{verbatim}
$ cnpath testPackage-1.0.0/resource1
<note: splitted for readability here in the manual>
/Users/stefano/Work/ GenomeAtlas/runApp/tests/testPackageDir1/
    testPackage-1.0.0.package/Resources/Linux-ia64/resource1
\end{verbatim}

