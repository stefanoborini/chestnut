\section{Setup of Chestnut Package Manager}

In order to use the Chestnut Package Manager, you have to perform two steps:
\begin{itemize}
\item put the path to the programs cnrun, cnpath and cnls in your
PATH variable, so that you can run them simply invoking the program name
\item define where the application should look for packages, defining
proper paths (even more than one) in the environment variable
PACKAGE\_SEARCH\_PATH.
\end{itemize}

This is done adding the following lines to your startup files:
\begin{verbatim}
in tcsh:
setenv PATH "${PATH}:/path/to/chestnut/bin"
setenv PACKAGE_SEARCH_PATH "/path/to/chestnut/packages/:$HOME/Packages"

in bash:
export PATH="${PATH}:/path/to/chestnut/bin"
export PACKAGE_SEARCH_PATH="/path/to/chestnut/packages/:$HOME/Packages"
\end{verbatim}

If you prefer to be able to override packages in the shared repository, you
can invert the two entries in PACKAGE\_SEARCH\_PATH, but in general this
operation is not needed.
