\section{Installation}
To install Chestnut Package Manager you need Python 2.4 or later installed on
your machine. Installation is very simple. Unpack the tar archive

\begin{verbatim}
$ tar xzvf Chestnut-2.2.0.tar.gz
\end{verbatim}

Then run the installation script

\begin{verbatim}
$ python setup.py install --home=$HOME
\end{verbatim}

Finally, set the following environment variables

\begin{verbatim}
$ export PATH="${PATH}:${HOME}/bin"
$ export CN_PACKAGE_SEARCH_PATH=$HOME/ChestnutPkgs:/opt/ChestnutPkgs
\end{verbatim}

It is suggested that you add these lines into your \verb+.bash_profile+, so that they
get parsed every time you login.  The first line makes sure you have
\verb+$HOME/bin+ in your \verb+PATH+ environment variable. This line could be
redundant, as this setting could already be present in your system. Adding it
twice does not really hurt.  After the installation is done, you will have
three executables in your \verb+$HOME/bin+ directory: \verb+cnrun+, \verb+cnls+
and \verb+cnpath+.  The second line specifies the directories where packages
will be searched. A possible choice for personal packages is
\verb+$HOME/ChestnutPkgs+. You can also have system-wide directories, for
example in \verb+/opt/ChestnutPkgs+, for example. Higher priority is given to
the first directories in the list, and lower priorities to the following ones
(exactly as in \verb+PATH+). You are free to add more if you want.

Finally you need to create the package directories. For the system-wide
directory, you need superuser privileges.  If you don't have superuser rights
on your system you can skip the creation of the system-wide directory and run
only the last command.

\begin{verbatim}
$ sudo mkdir -p /opt/ChestnutPkgs
$ sudo chmod 777 /opt/ChestnutPkgs
$ mkdir $HOME/ChestnutPkgs
\end{verbatim}

