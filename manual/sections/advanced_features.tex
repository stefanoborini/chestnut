\section{Advanced features}


\subsection{cnls advanced usage}

While using cnls, you can use globbing to list packages, but remember that you need to escape the symbols that could be interpreted by the shell.
\begin{verbatim}
$ cnls \*plot\*
roseplot-1.1.0
\end{verbatim}

When performing long listing with the \verb+-l+ switch, additional
information is printed in the form of one-letter case-sensitive flags.
The presence of:
    \begin{itemize}
    \item an \verb+x+ means that the package contains executables
    \item a \verb+r+ means that the package contains resources
    \item a \verb+d+ means that the package has a default entry point for an
    executable. This has consequences when you run it with cnrun.
    \item a \verb+m+ means that the package has executable entry points
    \textbf{other} than the default. With "other" we mean that if there's no
    default, this flag is present also if you have only one executable entry
    point.
    \item a \verb+D+ means that the package is deprecated or contains deprecated entries.
    \end{itemize}


\subsection{cnrun advanced usage }

It is possible to specify a package without or with a reduced definition of the
version: in this case, the most recent version will be chosen. For example.
having the following packages in our search path
\begin{verbatim}
foo-1.0.0
foo-1.0.1
foo-1.1.0
foo-1.2.0
foo-2.1.0
\end{verbatim}
\begin{itemize}
\item calling \verb+cnrun foo+ will run foo-2.1.0
\item calling \verb+cnrun foo-1+ will run foo-1.2.0
\item calling \verb+cnrun foo-1.0+ will run foo-1.0.1
\end{itemize}

This is useful to use the most recent compatible version as soon as it's
deployed, in particular when using the syntax \verb+cnrun foo-1+.  If you have
multiple packages with the same name, the one that gets executed is the first
that is found (honors the order specified in \verb+CN_PACKAGE_SEARCH_PATH+). 

cnrun is totally transparent in terms of input/output. you can use it in
unix pipelines, redirect the stdin or stdout, and pass options to the
invoked program. You can even run standard unix programs (although there's
apparently no real reason to do it, it is convenient to do so in some
advanced circumstances). If the package is not found, cnrun will try to
execute the program as in a standard shell PATH lookup. In other words, this
will work transparently
\begin{verbatim}
$ cnrun ls -la | cnrun less
\end{verbatim}

Existing packages will take precedence, so the presence of a ls-1.0.0 package,
for example, will execute this package instead of the \verb+ls+ command. Using
the absolute path will work in any case against the regular command. The
following will invoke \verb+ls+ even if a ls-1.0.0 package exists.
\begin{verbatim}
$ cnrun /bin/ls -la 
\end{verbatim}

\subsection{Bash autocomplete}

As of version 2.2.0, a bash autocomplete script is provided in the addon
directory. Simply source this script at startup to access useful autocompletion
features for your bash prompt, for example with the following syntax
\begin{verbatim}
. $HOME/chestnut_bash_completion 
\end{verbatim}
When using cnrun and cnpath, using Tab will complete the package name and the
entry point.
