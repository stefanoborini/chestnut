\section{Getting started: a quick overview}

\subsection{Basic usage}

Having completed the setup, we start using Chestnut for the first time. From the Chestnut sources,
copy the directory hello-1.0.0.package in your \verb+$HOME/ChestnutPkgs+ directory.

\begin{verbatim}
$ cd examples
$ cp -r hello-1.0.0.package $HOME/ChestnutPkgs/ 
\end{verbatim}
With this command you just deployed your first package. 
The command \verb+cnls+ displays a list of available packages.
\begin{verbatim}
$ cnls
hello-1.0.0
\end{verbatim}
If you do not obtain this result, your setup is incorrect.

You can get a little more information about what the program does by using the
\verb+-d+ switch
\begin{verbatim}
$ cnls -d
hello-1.0.0 : Prints a salutation message
\end{verbatim}
Or, if you are really curious, you can get even more information by using the \verb+-l+ switch
\begin{verbatim}
$ cnls -l
hello-1.0.0:
    Description: Prints a salutation message
    Flags: xd
    Location: /Users/stefano/ChestnutPkgs/hello-1.0.0.package
    Executable entry points:
        hello (default) : Prints hello!
\end{verbatim}

The \verb+cnrun+ program allows you to execute the hello package for the very first time
\begin{verbatim}
$ cnrun hello-1.0.0
hello!
\end{verbatim}

\subsection{Entry points and Resources}

To proceed further you will have to use the improved version hello-1.1.0, available in the examples directory.
\begin{verbatim}
$ cd examples
$ cp -r hello-1.1.0.package $HOME/ChestnutPkgs/ 
\end{verbatim}


Now you can check if the package is correctly installed by issuing

\begin{verbatim}
$ cnls -l hello-1.1.0
hello-1.1.0:
    Description: Contains utilities for various salutations
    Flags: xrdm
    Location: /Users/stefano/ChestnutPkgs/hello-1.1.0.package
    Executable entry points:
        bye  : Prints bye in the specified language
        hello (default) : Prints hello in the specified language
    Resource entry points:
        hello_translations : Translations for hello in various languages
        bye_translations : Translations for bye in various languages
\end{verbatim}
To proceed further, Chestnut terminology has to be explained. An entry point is
a named access into the package. If a package is a street, the entry point is
the house number. 

The above package has four entry points: bye, hello (which is set as the
default), hello\_translations and bye\_translations. The first two are executable
entry points, meaning that they are assigned to programs that can be executed
by means of \verb+cnrun+. The latter two are instead resource entry points,
meaning that they are assigned to files carrying no executable meaning, like
for example datafiles we want to make accessible. This package therefore
contains both executable programs and data.

We can execute this package as before, using \verb+cnrun+. If we do not specify
a specific entry point, the default one is used.  This time, the program can
provide salutation in different languages, by specifying the proper command
line argument. We can get it in italian for example
\begin{verbatim}
$ cnrun hello-1.1.0 it
Running for Darwin-Power Macintosh
Current CN_ROOT_DIR is /Users/stefano/ChestnutPkgs/hello-1.1.0.package
ciao (on Darwin-Power Macintosh)
\end{verbatim}
or in french
\begin{verbatim}
$ cnrun hello-1.1.0 fr
Running for Darwin-Power Macintosh
Current CN_ROOT_DIR is /Users/stefano/ChestnutPkgs/hello-1.1.0.package
salut (on Darwin-Power Macintosh)
\end{verbatim}
To invoke a different entry point of the same package, the following syntax is used
\begin{verbatim}
$ cnrun hello-1.1.0/bye it
Running for Darwin-Power Macintosh
Current CN_ROOT_DIR is /Users/stefano/ChestnutPkgs/hello-1.1.0.package
ci vediamo (on Darwin-Power Macintosh)

$ cnrun hello-1.1.0/bye fr
Running for Darwin-Power Macintosh
Current CN_ROOT_DIR is /Users/stefano/ChestnutPkgs/hello-1.1.0.package
au revoir (on Darwin-Power Macintosh)
\end{verbatim}
In the above examples the program associated to the \verb+bye+ entry point is now running.

If a default executable entry point is present, the package is said to be
executable. If a package does not have a default executable entry point, then
an entry point must always be specified. A package can also contain
no executable entry points at all (for example, for packages containing just resources).

Resources have entry points as well, and they can used to obtain the current
absolute path of the resource by means of the program \verb+cnpath+ (long line
wrapped for presentation reasons) 
\begin{verbatim}
$ cnpath hello-1.1.0/hello_translations
/Users/stefano/ChestnutPkgs/hello-1.1.0.package/ \ 
 Resources/Darwin-Power Macintosh/hello_translations
\end{verbatim}

\subsection{Relocation}

Chestnut helps you with minimizing the disruption of relocating packages
around.  If you move the hello package into another repository, for
example in \verb+/opt/ChestnutPkgs+, everything will still work, as far as
\verb+CN_PACKAGE_SEARCH_PATH+ includes this directory as well.

\begin{verbatim}
$ mv $HOME/ChestnutPkgs/hello-1.0.0.package /opt/ChestnutPkgs/
$ cnrun hello-1.0.0
hello!
$ cnls -l
hello-1.0.0:
    Description: Prints a salutation message
    Flags: xd
    Location: /opt/ChestnutPkgs/hello-1.0.0.package
    Executable entry points:
        hello (default) : Prints hello!
\end{verbatim}

You can see from the Location line that the package is now in another
directory.  As said, Chestnut acts as the shell PATH variable: PATH removes the
burden of knowing the absolute location of an executable on the filesystem.
Chestnut removes the burden of knowing the absolute location of a whole package
on the filesystem. Both require you to specify "magic directories" where this
mechanism is actuated.

The same result will be obtained for resources and the output of \verb+cnpath+
\begin{verbatim}
$ mv $HOME/ChestnutPkgs/hello-1.1.0.package /opt/ChestnutPkgs
$ cnpath hello-1.1.0/hello_translations
/opt/ChestnutPkgs/hello-1.1.0.package/ \
  Resources/Darwin-Power Macintosh/hello_translations
\end{verbatim}

