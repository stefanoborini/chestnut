\section{Getting started: a quick overview}

\subsection{Basic usage}

Having completed the setup, we start using Chestnut for the first time. From the Chestnut sources,
copy the directory hello-1.0.0.package in your \verb+$HOME/ChestnutPkgs+ directory.

\begin{verbatim}
cd Examples
cp -r hello-1.0.0.package $HOME/ChestnutPkgs/ 
\end{verbatim}

With this command you just deployed your first package. 

The command \verb+cnls+ displays a list of available packages.

\begin{verbatim}
$ cnls
hello-1.0.0
\end{verbatim}

If you do not obtain this result, your setup is incorrect.

You can get a little more information about what the program does by using the
\verb+-d+ switch

\begin{verbatim}
$ cnls -d
hello-1.0.0 : Prints a salutation message
\end{verbatim}

Or, if you are really curious, you can get even more information by using the -l switch

\begin{verbatim}
$ cnls -l
hello-1.0.0:
    Description: Prints a salutation message
    Flags: xd
    Location: /Users/stefano/Packages/hello-1.0.0.package
    Executable entry points:
        hello (default) : Prints hello!
\end{verbatim}

The \verb+cnrun+ program allows you to execute the hello package for the very first time

\begin{verbatim}
$ cnrun hello-1.0.0
hello!
\end{verbatim}

\subsection{Entry points and Resources}

To proceed further you will have to download an improved version of hello, from
here.

Once you obtain the tar.gz file, unpack it so that the result is produced into
a proper location where Chestnut can find it, for example by issuing

\begin{verbatim}
tar -xzv -C $HOME/Packages -f hello-1.1.0.package.tar.gz
\end{verbatim}

Now you can check if the package is correctly installed by issuing

\begin{verbatim}
$ cnls -l hello-1.1.0
hello-1.1.0:
    Description: Contains utilities for various salutations
    Flags: xrdm
    Location: /Users/stefano/Packages/hello-1.1.0.package
    Executable entry points:
        bye  : Prints bye in the specified language
        hello (default) : Prints hello in the specified language
    Resource entry points:
        hello_translations : Translations for hello in various languages
        bye_translations : Translations for bye in various languages
\end{verbatim}
Here things are different. We will now quickly introduce the terminology of
Chestnut. 

An entry point is a named access into the package. If a package is a
street, the entry point is the house number. Invoking \verb+cnrun+
as done before, the entry point was not specified, and the default was used.

The above package has four entry points: bye, hello (which is set as the
default), hello\_translations and bye\_translations. The first two are executable
entry points, meaning that they are assigned to programs that can be executed
by means of \verb+cnrun+. The latter two are instead resource entry points,
meaning that they are assigned to files carrying no executable meaning, like
for example datafiles we want to make accessible. This package therefore
contains both executable programs and data.

We can execute this package as before, using \verb+cnrun+. This
time, the program can provide salutation in different languages, by specifying
the proper command line argument. We can get it in italian for example

\begin{verbatim}
$ cnrun hello-1.1.0 it
Running for Darwin-Power Macintosh
Current PACKAGE_ROOT_DIR is /Users/stefano/Packages/hello-1.1.0.package
ciao (on Darwin-Power Macintosh)
\end{verbatim}
or in french
\begin{verbatim}
$ cnrun hello-1.1.0 fr
Running for Darwin-Power Macintosh
Current PACKAGE_ROOT_DIR is /Users/stefano/Packages/hello-1.1.0.package
salut (on Darwin-Power Macintosh)
\end{verbatim}
With this invocation procedure we are currently executing the default executable entry point: \verb+hello+. 
We can execute a different entry point with the following syntax

\begin{verbatim}
$ cnrun hello-1.1.0/bye it
Running for Darwin-Power Macintosh
Current PACKAGE_ROOT_DIR is /Users/stefano/Packages/hello-1.1.0.package
ci vediamo (on Darwin-Power Macintosh)
\end{verbatim}

\begin{verbatim}
$ cnrun hello-1.1.0/bye fr
Running for Darwin-Power Macintosh
Current PACKAGE_ROOT_DIR is /Users/stefano/Packages/hello-1.1.0.package
au revoir (on Darwin-Power Macintosh)
\end{verbatim}

As you can see, a different program is running now, the one associated to the \verb+bye+ entry point.

If a default executable entry point is present, the package is said to be
executable. If a package does not have a default executable entry point, then
an entry point must always be specified. A package can also contain
no executable entry points at all (for example, for packages containing just resources).

Resources have entry points as well, and they can used to obtain the current
absolute path of the resource by means of the program \verb+cnpath+

\begin{verbatim}
$ cnpath hello-1.1.0/hello_translations
/Users/stefano/Packages/hello-1.1.0.package/Resources/Darwin-Power Macintosh/hello_translations
\end{verbatim}

\subsection{Relocation}

Now, let’s try to move the hello package into another Package repository, for example in /opt/Packages.

$ mv $HOME/Packages/hello-1.0.0.package /opt/Packages/
$ cnrun hello-1.0.0
hello!

As you can see, everything works as before, but if you try to invoke cnls, this
time will tell you a different story about the location of the package

$ cnls -l
hello-1.0.0:
    Description: Prints a salutation message
    Flags: xd
    Location: /opt/Packages/hello-1.0.0.package
    Executable entry points:
        hello (default) : Prints hello!

The package is now in another directory. We are not concerned about where the
package is located when invoking cnrun. We are only concerned about the name
(and version) of the package we want to invoke. In this sense, Chestnut acts in
a similar way to the shell PATH variable: PATH removes the burden of knowing
the absolute location of an executable on the filesystem. Chestnut removes the
burden of knowing the absolute l ocation of a package on the filesystem. Both
require you to define “magic directories” where their mechanism is actuated.


If you happen to move the package somewhere else, you will get the correct absolute path of the resource

\begin{verbatim}
$ mv Packages/hello-1.1.0.package /opt/Packages/                   
$ cnpath hello-1.1.0/hello_translations
/opt/Packages/hello-1.1.0.package/Resources/Darwin-Power Macintosh/hello_translations
\end{verbatim}

