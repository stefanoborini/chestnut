\section{Quick reference}

\subsection{Listing}

\begin{itemize}
\item Getting the list of available packages
\begin{verbatim}
$ cnls
\end{verbatim}
\item Getting a long list of available packages
\begin{verbatim}
$ cnls -l
\end{verbatim}
\item Getting a long list of a specific package
\begin{verbatim}
$ cnls -d
\end{verbatim}
\item Getting a list with an additional description of the package
\begin{verbatim}
$ cnls -l package-1.0.0
\end{verbatim}
\item Getting a list of packages matching a globbing pattern
\begin{verbatim}
$ cnls \*foo\*
\end{verbatim}
\end{itemize}

\subsection{Running}

\begin{itemize}
\item Running a package default entry point
\begin{verbatim}
$ cnrun testPackage-1.0.0
$ cnrun testPackage-1.0
etc...
\end{verbatim}
\item Running a package alternative entry point
\begin{verbatim}
$ cnrun testPackage-1.0.0/secondaryEntryPoint
$ cnrun testPackage-1.0/secondaryEntryPoint
etc...
\end{verbatim}
\item Running a standard unix program 
\begin{verbatim}
$ cnrun ls
$ cnrun /bin/ls
\end{verbatim}
\end{itemize}

\subsection{Resources}

\begin{itemize}
\item Getting a path (resource or executable)
\begin{verbatim}
$ cnpath testPackage-1.0.0/resource1
\end{verbatim}
\end{itemize}
